\subsection{مدخل}
إن انتقال القيم هو عملية اكتساب الافراد للقيم الاجتماعية التي تتم  عبر قنوات تؤدي
فيها المؤسسات الاجتماعية الدور الرئيسي. ويمكن أن نصنف انتقال القيم في المجتمع عبر
المؤسسات إلى داخلية وخارجية.

\subsection{المؤسسات التي تنقل القيم من داخل المجتمع}
\begin{enumerate}
    \item\textbf{الاسرة:}
        هي أولى المؤسسات الاجتماعية التي تنقل القيم بطريقة مباشرة من الآباء الى
        الابناء حيث يتم تدريب الطفل على النظام الاجتماعي وعادات رمزية.
        \smallskip
        
        \textbf{مثال:}
        تحية الجيران التي الرمز الانفتاح على الجماعة.
        \smallskip
        
        \textbf{وهنا يمكن التمييز بين نوعين من الأسر}
        \begin{itemize}
            \item\textbf{الاسرة التقليدية:}
                التي تنقل القيم إلى أفرادها عن طريق التعليم والتلقين، فهي ما زالت
                متمسكة بتقاليدها الموروثة وتدرب الطفل على الطاعة والامتثال.
            \item\textbf{الاسرة الحديثة:}
                التي تدرب الطفل على روح المبادرة والابتكار واحترام الرأي والاختيار
                الفردي.
        \end{itemize}
    \item\textbf{المدرسة:}
        هي أولى المؤسسات النظامية وثاني المؤسسات الاجتماعية التي تنقل القيم بصفة
        غير شخصية، وتهدف إلى إكساب المتعلم المعرفة، وروح الانضمام إلى الجماعة 
        وتدربه على القواعد والنظام العام، وتعلمه الاحترام وكيفية التعامل مع
        الآخرين.
        \smallskip
        
        \textbf{وهنا يمكن التمييز بين نوعين من المدارس}
        \begin{itemize}
            \item\textbf{المدرسة التقليدية :}
                تهدف إلى اكساب المتعلم العلم والمعرفة بطريقة التلقين والحفظ ونقل
                علوم الأجيال السابقة واحترامها وتقليدها.
            \item\textbf{المدرسة الحديثة :}
                تعتمد على اطلاق روح الابتكار والمبادرة واكتشاف القواعد العلمية
                وإنشاء معرفة التلاميذ باستخدام ملكاتهم الفكرية والعلمية.
        \end{itemize}
    \item\textbf{الحي :}
        هو وسط اجتماعي أوسع من الاسرة ولكنه يجاورها، يكسب الطفل قيما اجتماعية
        جديدة من خلال الألعاب والمساعدة والشجاعة حيث يراعى فيه القواعد السلوكية
        المرغوب بها في المجتمع ويمارس فيه الطفل أولى خبراته الحياتية ويتعلم
        فيه التمييز بين الخاص والعام.
    \item\textbf{الدين :}
        يزود الفرد بقيم شاملة نابعة من الرسالة السماوية وتهديه إلى الخير وتنهاه
        عن الشر.
\end{enumerate}

\subsection{المؤسسات التي تنقل القيم من خارج المجتمع}
\begin{enumerate}
    \item\textbf{المدارس والجامعات :}
        إن الجامعات العلمانية منها وغير العلمانية تعتمد فلسفة تربوية نابعة من
        مجتمعاتها الأصلية. فهي تنقل قيم بلادها إضافة إلى القيم العلمية والفكرية
        وعندما يهاجر الطلاب إلى المجتمعات الغربية بحثا عن العلم يكتسبون القيم
        الغربية ويتبنونها وينقلونها عند عودتهم إلى مجتمعاتهم.
    \item\textbf{المؤسسات والشركات :}
        أدت إلى نشر القيم الاقتصادية الحديثة مع نمو قيم الربح والانتاج.
    \item\textbf{المقاهي والمسارح ودور السينما :}
        التي بدأت بنشر قيم فنية وترفيهية جديدة.
    \item\textbf{نمط العمارة في المجتمع المدني :}
        أدى إلى تبني قيم جديدة بعد الانتقال من الحارات الداخلية إلى الشوارع 
        الحديثة ذات المباني العالية والشقق المنفردة التي أدت إلى تغذية روح
        الانفصال عن الجماعة.
    \item\textbf{وسائل الاتصال :}
        تساهم في نقل القيم الغربية وتبث أفكارا جديدة ومشاهد عن مجتمعات بعيدة حيث
        يمكن ابراز أنماط مختلفة من العلاقات الاجتماعية ما قد يساعد على تقليدها
        وتبنيها.
\end{enumerate}

\subsection{تباين المواقف من القيم الخارجية (اختلاف المواقف)}
\begin{enumerate}
    \item\textbf{صراع القيم :}
        إن بعض المجتمعات تشهد صراعا بين القيم الوافدة والقيم المحلية نظرا للاختلاف
        فيها بينها وقد طال هذا الصراع الافكار والمعتقدات.
        \smallskip
        
        \textbf{مثال :}
        دخول مفهوم التحرر عبر المسلسلات التركية يعارض قيمة الحياء والعفاف في
        المجتمعات العربية ومنها لبنان ما يؤدي إلى بروز صراع القيم.
    \item\textbf{تعايش القيم وتفاعلها\footnotemark :}
        بعد دخول القيم الوافدة إلى المجتمع تتعايش وتتفاعل بالقدر الذي يمكن أن
        تقبل به في نظام القيم الأصيل. فتصبح هذه القيم الجديدة بعد تبنيها جزءا من
        نظام القيم.
        \smallskip
        
        \textbf{مثال :}
        عمل المرأة.
    \item\textbf{قيم إنسانية شاملة :}
        مثل حق الطفل والمرأة والمعوق في التعليم، حرية ابداء الرأي.
        
    \footnotetext{
        لو سئلنا عن مواقف المجتمع اتجاه القيم الوافدة، نقول إن هناك ثلاثة مواقف 
        ممكنة هي الصراع لو رفضت تلك القيم ووجدت على أنها مضرة للمجتمع، وتعايش لو
        ووجت على أنها غير مضرة ولا مفيدة، وتفاعل لو وجدت مفيدة.
    }
\end{enumerate}