\subsection{تعريف القيم}
هي أفكار ومبادئ ينبثق عنها أحكام ومُثل عليا تدفع الجماعة لتضع معايير لتحديد
    السلوك الأمثل. وهي تُضفي على السلوك الانساني معنىً خاصاً بها.

\subsection{خصائص القيم}
\begin{enumerate}
    \item\textbf{القيم متغيرة:}
        أي أنها قابلة للتبدل والتغير: إن القيم الأخلاقية أكثر تقبلا للتبدل للتغير
        لأنها قيم فردية، أما القيم الاجتماعية فمن الصعب ذلك لأنها تجمع على موافقة
        المجتمع نسبيا.
    \item\textbf{القيم مكتسبة:}
        يمكن اكتساب القيم ضمن وسط إجتماعي معين عبر وسائط التنشئة الاجتماعية.
    \item\textbf{القيم نسبية:}
        اختلاف القيم بين مجتمع وآخر وزمن وآخر وأهيمتها تختلف بين شخص وآخر.
    \item\textbf{القيم انتقالية:}
        انتقال القيم من جيل إلى آخر ومن الأقوى إلى الأضعف.
    \item\textbf{القيم انسانية:}
        خاصة مميزة للإنسان باعتباره كائنا عاقلا.
    \item\textbf{القيم مقدسة:}
        تتمتع القيم بقدسية كبيرة في المجتمع.
\end{enumerate}

\subsection{اختلاف النظرة حول تفسير مفهوم القيم}
\begin{enumerate}
    \item\textbf{منظور عالم الاجتماع:}
        أن القيم محصلة الجهود الفردية والاجتماعية المنسقة والمترابطة في المجتمع.
        وأن الوحدة الاجتماعية تتأمن بواسطة القيم المترسخة في الافراد.
    \item\textbf{منظور الفيلسوف:}
        أنها تنشأ من طبيعة الإنسان، وهي جزأ من فطرته العاطفية أو العقلية. وأن
        القيم نابعة من العقل.
    \item\textbf{منظور رجل الدين:}
        الاديان هي التي تهدي الانسان إلى القيم، فالرسالة الدينية هي مصدر القيم.
\end{enumerate}

\subsection{دراسة القيم}
\begin{enumerate}
    \item\textbf{اختبار القيم:}
        يعتبر علماء الاجتماع أن القيم موضوع قابل للاختبار والمقارنة والتحليل
        التجريبي لأن اختلاف المجتمعات وتباعدها يؤدي إلى تبنيها لقيم مختلفة.
    \item\textbf{تكون القيم:}
        تتكون القيم في البيئة الاجتماعية التي يعيش فيها الانسان ومن خلال ظروف
        الحياة، وهذه الظروف تستدعي من الجماعة أن تحدد الأنظمة التي تنظم أمور
        الجماعة وشؤونها (النظام الاقتصادي - الاجتماعي - السياسي - التربوي -
        السلوكي).
        وتكون عاملا مهما من عوامل وحدة الجماعة، وهذا ما يفسر اختلاف القيم بين
        مجتمع وآخر.
\end{enumerate}

\subsection{وظائف القيم}
\begin{enumerate}
    \item\textbf{تماسك النموذج:}
        هي سلسلة من القواعد المترابطة فيما بينها لإظهار نموذج سلوكي غايته تحقيق
        المثل الاعلى للاقتراب من القيمة.
        \medskip
        
        \textbf{مثال عن الاحتشام:}
        لتطبيق قيمة الاحتشام يجب الاحتشام في الملبس والكلام والسلوك والتعامل...
        فلا يمكن في الملبس وعدمه في الكلام.
        \medskip
        
        \textbf{مثال عن السلوك الديمقراطي:}
        لتطبيق السلوك الديمقراطي يقتضي مواقف ديموقراطية في الأسرة والعمل والحياة
        الاجتماعية والسياسية فلا يمكن تطبيقه في الحياة السياسية وعدمه مع الزوجة
        والأولاد أو في العمل.
    \item\textbf{وحدة الاشخاص النفسية:}
        إن تبني الاشخاص للقيم ذاتها يساهم في توحدهم وتجمعهم. ويجعل مسافة التواصل
        بينهم واسعة. كما ويجعل العلاقة بين عناصرها منسجمة في الرؤية والموقف نظرا
        لتبنيها القيم ذاتها. قد تتجلي وحدة الاشخاص النفسية في إطار الفرد للجماعة
        أو الجماعة للفرد.
        \medskip
        
        \textbf{مثال عن المراهقين:}
        إن المراهقين يؤلفون جماعتهم الخاصة التي تتميز بقيم وسلوكيات خاصة بهم
        والتي قد تختلف عن قيم وسلوكيات المجتمع.
        \medskip
        
        \textbf{مثال في المجتمعات الغربية:}
        إن القيم السائدة عن الحرية والعدالة والديمقراطية تجعل جميع الاشخاص يتبنون
        الموقف ذاته من قضية ضرب الأطفال مثلا.
        \medskip
        
        \textbf{مثال في المجتمعات التقليدية العصبية:}
        إن الاهانة للفرد تعتبر موجهة للجماعة بكاملها.
    \item\textbf{التكامل الاجتماعي:}
        هي عملية التنسيق بين مختلف الافراد والجماعات المختلفة وغيرها من أنماط
        المجتمع في وحدة متكاملة من حيث الادوار والوظائف.
        \smallskip
        
        \textbf{مثال:}
        إن القيم تجعل الافراد متكاملين فالشخص الذي يعجز عن تأمين شيء ما فإن القيم
        تدفع الآخرين إلى متابعة وتكملة ما عجز عنه هذا الفرد.
        \smallskip
        
        \textbf{مثال:}
        في المجتمعات التقليدية كان بناء المنزل يتم من خلال مشاركة جميع أبناء
        القرية، أو دفع الدية لأهل القتيل يجمع من جميع أبناء القرية.
\end{enumerate}

\subsection{أنواع القيم}
\begin{enumerate}
    \item\textbf{القيم الاجتماعية:}
        هي المثل العليا التي يفضلها أو يرغب فيها الناس في ثقافة معينة ذات صفة
        العمومية بالنسبة لجميع الأفراد، كما تصبح من موجهات السلوك تعتبر هدفا له
        مثل: العيش المشترك والاعتراف بالآخر.
    \item\textbf{القيم الاختلاقية:}
        هي ترسم معايير الخير والشر وتبين لنا متى يكون الفعل خيرا أو شرا، وهي
        نسبية تختلف من شخص لآخر.
    \item\textbf{القيم الانسانية الشاملة:}
        هي قيم تبنتها أكثر المجتمعات ونادت بها الجمعيات والقوانين الدولية وشرعية
        حقوق الانسان.
        \smallskip
        
        \textbf{مثال:}
        الحق في العلم، العمل، الحريات، حقوق البيئة، العدالة، حقوق المرأة، حقوق
        الطفل بعد تطور التواصل بين المجتمعات.        
\end{enumerate}