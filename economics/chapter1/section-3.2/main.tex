\documentclass[a4paper]{article}
\usepackage[margin=1in]{geometry}

\usepackage{polyglossia}
\setdefaultlanguage{arabic}
\setmainfont{Amiri}

\setlength{\parindent}{0pt}

\begin{document}

\begin{center}
\huge{تابع التطبيقات من الدرس الثالث}
\end{center}

\section{تطبيقات}

\begin{enumerate}
    \setcounter{enumi}{3}
    \item أجب بصح أم خطأ مبررا إجابتك:
        \begin{enumerate}
            \item عند ارتفاع الضرائب غير المباشرة تنخفض معدلات البطالة.
            \item إن زيادة النفقات على الابحاث تساهم في زيادة القدرة التنافسية 
                لدى المؤسسات.
            \item إن زيادة العجز في الميزان المدفوعات يساهم في ارتفاع الدين
                العام.
            \item إن حالة النمو غير المتوازن على صعيد القطاعات تساهم في تفاقم
                العجز في الميزان التجاري.
            \item يساهم التوظيف غير المجدي في المؤسسات العامة في ارتفاع النمو
                الاقتصادي.
            \item إن هروب الرساميل الأجنبية يساهم في ارتفاع قيمة العملة الوطنية.
            \item عند تطبيق الحركية الاجتماعية تتراجع حدة التفاوت في المداخيل.
        \end{enumerate}
\end{enumerate}

\section{أجوبه}

\end{document}
